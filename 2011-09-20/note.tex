\documentclass[12pt]{article}
\usepackage{geometry}
\usepackage{amsmath}
\usepackage{amsthm}
\usepackage{amssymb}
\usepackage{mathrsfs}
\usepackage{parskip}
\usepackage{enumerate}
\usepackage{stmaryrd}
\usepackage{listings}
\usepackage{fullpage}


% command definitions
\newcommand{\Scr}[1]{\ensuremath{\mathcal{#1}}}
\newcommand{\Lang}{\ensuremath{\Scr{L}^p}}
\newcommand{\Atom}{\ensuremath{\text{Atom}(\Lang)}}
\newcommand{\Form}{\ensuremath{\text{Form}(\Lang)}}
\newcommand{\Nat}{\ensuremath{\mathbb{N}}}
\newcommand{\Valu}{\ensuremath{ t:\Atom \to \{0,1\}}}
\newcommand{\limplies}{\ensuremath{\to}}
\newcommand{\ltauteq}{\ensuremath{\models\!\!\!|}}
\newcommand{\liff}{\ensuremath{\leftrightarrow}}
\newcommand{\Term}{\ensuremath{\text{Term}(\Scr{L})}}
\newcommand{\AtomL}{\ensuremath{\text{Atom}(\Scr{L})}}
\newcommand{\FormL}{\ensuremath{\text{Form}(\Scr{L})}}
\newcommand{\Temp}{\ensuremath{\text{Temp}(\Scr{L})}}
\newcommand{\Cond}[1]{\ensuremath{\,\llparenthesis\,#1\,\rrparenthesis\,}}
\newcommand{\Group}[1]{\ensuremath{\langle #1 \rangle}}

\begin{document}

\title{CS 348 Notes}
\author{Matthew Visser}
\date{Sep 20, 2011}
\maketitle

\section{The Relational Model}

\textbf{Main Idea:}  All information is organized in (flat) relations.

A row in a table is a relation between its fields, and where it is stored.

Consider \texttt{Student( Num, Name, Age )}. If we want to define that no two
students can have the same number and no student can have the same name:
\begin{equation}
    \forall x_1, y_1 x_2, y_2 x_3, y_3 ( ( Student (x_1,y_1,x) \lor
    Student(x_2,y_2,x) \lor y_1 = y_2) \limplies (x_1 = x_2))
\end{equation}


\subsection{Formal Definition}

\begin{description}
    \item[Universe:] A set of atomic values $D$ with equality (=).
    \item[Domain]  A name $D$ with a set of values $\text{dom}(D) \subseteq D$
    \item[Relation] 
        \begin{description}
            \item[Schema:] $R(A_1:D_1,A_2:D_2,\dots,A_k:D_k)$ with
                \begin{itemize}
                    \item name $R$
                    \item $A_1, \dots A_k$ a set of distinct attribute names
                    \item $D_1,\dots,D_k$ a collection of (net necessarily distinct)
                        domain names
                \end{itemize}
            \item[Instance:] A \textit{finite} relation $R\subseteq
                \text{dom}\left( D_1 \right)\times \dots \times \text{dom}\left(
                D_k \right)$
        \end{description}
\end{description}

\textit{See slide 4 of \texttt{relmodel-handout.pdf}}.

\subsection{Properties}

\begin{enumerate}
    \item Based on set theory
    \item All attribute values are atomic -- 
    \item Degree (arity) is the number of attributes in the schema
    \item Cardinality is the number of tuples in an instance.
\end{enumerate}

This means that once you get a column value, you are as deep as you can get in
terms of relations.

\textit{Example database on slide 6 of \texttt{relmodel-handout.pdf}}.

\subsection{SQL Vs Relational Model}

\begin{enumerate}
    \item Semantics of instances
        \begin{itemize}
            \item Relations are sets of tuples
            \item Tables are multisets of tuples
        \end{itemize}
    \item Unknown values
        \begin{itemize}
            \item SQL can have values as null.
        \end{itemize}
\end{enumerate}

\subsection{Integrity Constraints}

Extend the relational model with constraints on data. A schema is only valid if
it meets all constraints.

You would want to do this if
\begin{enumerate}
    \item You want to ensure the data is as expected and as designed. Closer to
        what you intended than what can happen.
    \item Protect from bugs
\end{enumerate}

There are different types of constraints
\begin{itemize}
    \item Tuple-level
        \begin{itemize}
            \item Domain restructions, attribute comparisons
        \end{itemize}
    \item Relation-level
        \begin{itemize}
            \item Key constraints -- can be determined by a row in a table.
                \begin{itemize}
                    \item Super key: A subset of columns such that no two rows
                        will have the same values of these columns. An example
                        is that in a set of rows, no row will have all columns
                        the same.
                    \item Candidate key: A minimal superkey. The minimal set of
                        columns that uniquely identify a row.
                    \item Primary key: A candidate key that uniquely identifies
                        a row. Used for joins, referencing a
                        column,\textit{etc.} Expect that it is never null and is
                        immutable.
                \end{itemize}
            \item Functional Dependencies
        \end{itemize}
    \item Database-level
        \begin{itemize}
            \item Referential integrity
                \begin{itemize}
                    \item Foreign key: primary key of one relation appearing as
                        an attribute in another relation
                    \item Referential integrity: a tuple with a non-null value
                        for a foreign key that does not match the primary key
                        value of a tuple in the referenced relation isn't
                        allowed
                \end{itemize}
            \item Inclusion dependencies.
                $R(A_1,\dots,A_k \subseteq S(B_1,\dots,B_k)$
        \end{itemize}
\end{itemize}

\section{Relational Algebra}

\begin{itemize}
    \item Consists of a set of \textit{operators}.
    \item it is \textit{closed}
    \item Notation: $R$ is a relation name; $E$ is an expression.
\end{itemize}

\textit{To be continued\dots}


\end{document}
% vim: tw=80
