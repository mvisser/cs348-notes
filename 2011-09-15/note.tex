\documentclass[12pt]{article}
\usepackage{geometry}
\usepackage{amsmath}
\usepackage{amsthm}
\usepackage{amssymb}
\usepackage{mathrsfs}
\usepackage{parskip}
\usepackage{enumerate}
\usepackage{stmaryrd}
\usepackage{listings}
\usepackage{fullpage}

\begin{document}

\title{CS 348 Notes}
\author{Matthew Visser}
\date{Sep 15, 2011}
\maketitle

\section*{Schemas}
\subsection*{3-Level Schema Architecture}

\begin{enumerate}
    \item External Schema (view): what the applications see.
    \item Conceptual Schema: Description of structure of \textit{all} data in
        database.
    \item Physical schema: Description of physical aspects (files, devices,
        algorithms, etc.)
\end{enumerate}

\begin{description}
    \item[Database Instance:] A database with real data that conforms to a
        schema.
\end{description}

See slide 14 of \texttt{overview-present.pdf}.

\subsection*{Data Independence}

There are 2 kinds:
\begin{description}
    \item[Physical:] immune to changes in storage structure.
    \item[Logical:] immune to changes in data organization.
\end{description}

\subsection*{Interfacing to DBMS}

\begin{description}
    \item[DataDefinition Language (DDL):] specifies schemas.
    \item[Data Manipulation Language (DML):] Specifies queries and updates.
        \begin{itemize}
            \item navigational (procedural): application logic needs to navigate
                data storage.  This is like \textit{Propositional Logic}.
            \item non-navigational (declarative): Logical relations and have
                quantification. This is in \textit{First Order Logic}. An
                example of quantification is the SQL \texttt{FROM} keyword.
        \end{itemize}
\end{description}

\subsection*{Important Items}
\begin{itemize}
    \item Physical data independence
    \item Can have non-navigational languages.
\end{itemize}

\section*{Database Users}

\begin{description}
    \item[End User]\hfill
        \begin{itemize}
            \item Accesses database indirectly through application
            \item Generates ad-hoc queries using DBML
        \end{itemize}
    \item[Application Developer]\hfill
        \begin{itemize}
            \item Design and develop applications that access database
        \end{itemize}
    \item[Database Administrator (DBA)]\hfill
        \begin{itemize}
            \item manages conceptual schema
            \item Assist with application view integration
            \item monitors and tunes performance
            \item loads, reformats, populates data
        \end{itemize}
\end{description}

\section*{Transactions}

\textbf{Key Idea}: Ever application may think it is the only application
accessing data. This complicates things, because when things run concurrently,
mistakes could be made.

\begin{description}
    \item[Transaction:] A \textbf{Transaction} is an application-specified
        atomic and durable unit of work.
\end{description}

The properties of a transaction ensured by a DBMS are \textbf{ACID}:
\begin{description}
    \item[Atomic] -- A transaction occurs completely or not at all.
    \item[Consistent] -- each transaction preserves the consistency of the
        database
    \item[Isolated] -- one transaction does not interfere with another.
    \item[Durable] --  once completed, the changes are permanant. This ensures
        reliability.
\end{description}


\subsection*{Three main ideas}

\begin{itemize}
    \item Data independence -- data is organized properly.
    \item Transactions -- data integrity, allows concurrency.
    \item Quantification -- don't need to know internal structure.
\end{itemize}


\end{document}
% vim: tw=80
