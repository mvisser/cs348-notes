\documentclass[12pt]{article}
\usepackage{geometry}
\usepackage{amsmath}
\usepackage{amsthm}
\usepackage{amssymb}
\usepackage{mathrsfs}
\usepackage{parskip}
\usepackage{enumerate}
\usepackage{stmaryrd}
\usepackage{listings}
\usepackage{fullpage}

\begin{document}

\title{CS 348 Notes}
\author{Matthew Visser}
\date{Oct 27, 2011}
\maketitle

\section{E-R Modelling}

\begin{itemize}
	\item Types of attributes that usually should be made into an entity:
		\begin{itemize}
			\item A composite attribute has a set of entities which may or may not be
				defined.
			\item A multi-valued attribute is an attribute with which an entity can have
				a finite number.
		\end{itemize}
	\item Specialization and subsets of entities is described on slide 21 of
		\texttt{ermodel-handout.pdf}.
	\item Generalization is almost the same as specialization, but in plural.
\end{itemize}


\end{document}
% vim: tw=80
