\documentclass[12pt]{article}
\usepackage{geometry}
\usepackage{amsmath}
\usepackage{amsthm}
\usepackage{amssymb}
\usepackage{mathrsfs}
\usepackage{parskip}
\usepackage{enumerate}
\usepackage{stmaryrd}
\usepackage{listings}

\begin{document}

\newtheorem{mydef}{Definition}

\title{CS 348 Notes}
\author{Matthew Visser}
\date{Sep 13, 2011}
\maketitle

\section{Functional Dependancies and Normal Forms}

\section{Normal Forms}

A relation is in BCNF if for every FD $X \to A \in F$
\begin{itemize}
	\item $X \to A$ is trivial, \textit{i.e.} $A \in X$ \emph{or}
	\item $X$ is a superkey, \textit{i.e.}, $X \to R$.
\end{itemize}

A relation is in 3NF if for every FD $X \to A \in F$
\begin{itemize}
	\item $X \to A$ is trivial, \textit{i.e.} $A \in X$ \emph{or}
	\item $X$ is a superkey, \textit{i.e.}, $X \to R$.
	\item $A$ is part of some key for $R$.
\end{itemize}

\subsection{Minimal Cover}

\begin{enumerate}
	\item Put each FD in standard form, \textit{i.e.}, $X \to A$ where $A$ is a
		single attribute.
	\item Minimize the left side. If we have a rule $X \to A$ where $X$ has
		multiple attributes, if $B \in X$ and $X \setminus B \to B$ then remove
		$B$ from $X$.
	\item For the remaining FDs, $G$, see if one can be deleted without having
		the closure of our minimal cover, $G^+ \neq F^+$.
\end{enumerate}

\subsection{BCNF Decomposition}

\begin{enumerate}
	\item For $R$ not in BCNF, let $X \subset R$, $A$ be a single attribute in
		$R$, and $X \to A$ be an FD that causes a violation of BCNF. Decompose
		$R$ to $R \setminus A, XA$.
	\item If $R\setminus A$ or $XA$ is not in BCNF, recurse on that relation.
\end{enumerate}

\subsection{3NF Decomposition}

We can use the same algorithm as BCNF. If we want something dependency
preserving, we can do this.
\begin{enumerate}
	\item Do the BCNF decomposition algorithm.
	\item Find the minimal cover of each relation.
	\item Identify the set $N$ of dependencies in $F$ that is not preserved,
		\textit{i.e.}, not included in the closure of the union of the FD's of
		each sub-relation.
	\item For each FD $(X \to A) \in N$, create a relation schema $XA$ and add
		it to the decomposition of $R$.
\end{enumerate}

\end{document}
% vim: tw=80
