\documentclass[12pt]{article}
\usepackage{geometry}
\usepackage{amsmath}
\usepackage{amsthm}
\usepackage{amssymb}
\usepackage{mathrsfs}
\usepackage{parskip}
\usepackage{enumerate}
\usepackage{stmaryrd}
\usepackage{listings}
\usepackage{fullpage}

\begin{document}

\title{CS 348 Notes}
\author{Matthew Visser}
\date{Oct 13, 2011}
\maketitle

I missed the part about triggers. Oh well.

\section{SQL Application Development}

\subsection{Embedded SQL}

The application is put through a SQL preprocessor, then put through the C
compiler.

The application has three possible steps for a transaction:
\begin{enumerate}
	\item start transaction (implicit)
	\item commit --- once the work is done, write it
	\item abort --- if there was an error then don't commit and put everything
		back the way it was.
\end{enumerate}

\emph{Host variables} are used to send and receive values from the database
system. These are variables that the SQL language shares with the C language.
Information can be sent and recieved using these host variables.

The host variables correspond to a SQL type.

What about \texttt{NULL}? They are represented by a special return code (100).

\subsection{Cursors in Embedded SQL}

The cursor has a pointer to a tuple. You can use this to iterate results.




\end{document}
% vim: tw=80
