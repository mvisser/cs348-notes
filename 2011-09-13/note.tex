\documentclass[12pt]{article}
\usepackage{geometry}
\usepackage{amsmath}
\usepackage{amsthm}
\usepackage{amssymb}
\usepackage{mathrsfs}
\usepackage{parskip}
\usepackage{enumerate}
\usepackage{stmaryrd}
\usepackage{listings}

\begin{document}

\newtheorem{mydef}{Definition}

\title{CS 348 Notes}
\author{Matthew Visser}
\date{Sep 13, 2011}
\maketitle

\section*{Intro}

\textbf{URL}: http://db.uwaterloo.ca/~gweddell/cs348

\textbf{Newsgroup}: uw.cs.cs348

\textbf{Textbook}: \textit{Database Management Systems}, Raghu Ramakrishman and
Johannes Gehrke.(3rd ed. preferred).

\textbf{Assignments}:
\begin{enumerate}
    \item SQL
    \item C \& SQL (might be team projects)
    \item mystery
\end{enumerate}

Programming will be largely in SQL. Will be using DB2.

\textbf{Grad fair}: Tues Sept 20, from 1000h - 1400h in SLC.

\textbf{CS Grad Studies Info Session}: Wed Sept 21, 1200h - 1300h in DC 1302.

\textbf{Masters in Health Informatics Info Session}: Wed Oct 5th, 1200h - 1300h
in DC 1304.

\section*{Content}

\subsection*{Relational Model}

\begin{itemize}
    \item Theoretical / Foundational (i.e. First Order Logic)
    \item Standards (i.e. SQL)
        \begin{itemize}
            \item exists since 1989
        \end{itemize}
    \item Technology (i.e. particular products)
\end{itemize}

\subsection*{Use}

\begin{mydef}
    A \textbf{Database} is a large and persistent collection of pieces of
    information organized in a way that facilitates efficient retrieval and
    modification. \newline
\end{mydef}

\begin{mydef}
    A \textbf{Database Management System (DBMS)} is a program (or set of
    programs) that manages details related to storage and access for a database.
    \newline
\end{mydef}

\end{document}
% vim: tw=80
