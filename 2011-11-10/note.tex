\documentclass[12pt]{article}
\usepackage{geometry}
\usepackage{amsmath}
\usepackage{amsthm}
\usepackage{amssymb}
\usepackage{mathrsfs}
\usepackage{parskip}
\usepackage{enumerate}
\usepackage{stmaryrd}
\usepackage{listings}
\usepackage{fullpage}

\begin{document}

\title{CS 348 Notes}
\author{Matthew Visser}
\date{Nov 10, 2011}
\maketitle

\section{Functional Dependencies}

\textbf{What is the order of $F+$?}

Say we have relation $R$ with attributes $\{A_1,\dots,A_n\}$.

Consider $F = \{A1\to A_1\dots A_n\}$. $\implies (A_1 \to Z) \in F^{+}$ for any
subset $Z$ of $\{A_1,\dots,A_n\}$.

Therefore the order of $F^+ \in \Omega(e^n)$.

\begin{enumerate}
	\item $X \to Y$
	\item Does only allowing $X \to Y$ reduce expressiveness?
	\item Does only allowing $A,B\to C$ or $A \to C$ reduce expressiveness?
\end{enumerate}

\section{Decompositions}

We want a decomposition to take a set of attributes and break them up into
relations that are a subset of those attributes. We want this to not:
\begin{itemize}
	\item Lose information
	\item complicate checking of constraints
	\item contain anomalies (or at least we want it to contain fewer of those)
\end{itemize}

A decomposition is lossy unless
\[
R_1 \cap R_2 \to R_1\]
or
\[
R_1 \cap R_2 \to R_2\]

A decomposition is better for dependency preservation if we can check all of the
constraints without joining the tables. If we have to join then we call that
constraint an \emph{interrelational constraint}.

\section{Normal Forms}

We want something anomaly free and nonredundant.

\subsection{BCNF}

Schema $R$ is in BCNF (w.r.t. $F$) iff whenever $(X \to Y) \in F^+$ and
$XY \subseteq R$, then either
\begin{itemize}
	\item $(X \to Y)$is trivial \textit{i.e.}, $Y \subseteq X$ or
	\item $X$ is a superkey of $R$.
\end{itemize}

Why does this avoid redundancy?

It splits tables so that we have tables with keys that joined together can give
the original.

It is possible to have a functional dependancy that does not allow for BCNF.

\subsection{Third Normal Form}

A schema $R$ is in 3NF iff whenever $(X \to Y \in F^+)$ and $XY \subseteq R$,
then either
\begin{itemize}
	\item $(X \to Y)$ is trivial
	\item $x$ is a soperkey of $R$ or 
	\item each attribute in $Y\setminus X$ is contained in a candidate key of
		$R$.
\end{itemize}

This definition is looser than BCNF because it allows redundancy.

\end{document}
% vim: tw=80
