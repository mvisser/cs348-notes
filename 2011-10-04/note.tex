\documentclass[12pt]{article}
\usepackage{geometry}
\usepackage{amsmath}
\usepackage{amsthm}
\usepackage{amssymb}
\usepackage{mathrsfs}
\usepackage{parskip}
\usepackage{enumerate}
\usepackage{stmaryrd}
\usepackage{listings}
\usepackage{fullpage}

\begin{document}

\title{CS 348 Notes}
\author{Matthew Visser}
\date{Oct  4, 2011}
\maketitle

\section{Subqueries}

Subqueries can return multiple data sets to filter by, \textit{etc}.

They can also be used in a \texttt{with} clause (slide 23).

\section{Outer Joins}

Using a \texttt{left outer join} is essentially just syntactic sugar. Can be
written as a union. Essentially contains all records from left table.

Query on slide 24 equivalent to:
\begin{verbatim}
( select deptno, deptname, null as lastname
  from department
  where mgrno is null
  and deptno like 'D%' )
union
( select deptno, deptname, lastname
  from department d
       employee e
  where e.empno = d.empno
  and deptno like 'D%' )
\end{verbatim}

\subsection{Groups and Aggregates}

Can sort with the \texttt{group by} clause.

The \texttt{having} clause can filter by aggregate functions.  If you have a
group by and are sorting by aggregates and want to filter them out, you can do
this.

\end{document}
% vim: tw=80
