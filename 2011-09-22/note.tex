\documentclass[12pt]{article}
\usepackage{geometry}
\usepackage{amsmath}
\usepackage{amsthm}
\usepackage{amssymb}
\usepackage{mathrsfs}
\usepackage{parskip}
\usepackage{enumerate}
\usepackage{stmaryrd}
\usepackage{listings}
\usepackage{fullpage}

\begin{document}

\title{CS 348 Notes}
\author{Matthew Visser}
\date{Sep 22, 2011}
\maketitle

\section{Relational Algebra}

\subsection{Operators}

Look at slides.

\subsection{Selection}

Returns a subset of tuples from the entity.

\subsection{Projection}

Returns a subset of attributes from an expression.

\subsection{Rename}

Renames attributes in the list.

\subsection{Product}

\begin{itemize}
    \item Result has a tuple for every distinct pair of tuples in the two
        expressions.
    \item Need to rename when $E_1$ and $E_2$ have collisions in attribute
        names \textit{i.e.} can't chare attributes.
\end{itemize}

\subsection{Join}

\begin{itemize}
    \item The same as a cross product, then selection, \textit{i.e.}
        \[
        E_1 \Join_{condition} E_2 = \sigma_{condition}(E_1 \times E_2)
        \]
    \item Natural join uses the implied constraint on foreign keys.
\end{itemize}



\end{document}
% vim: tw=80
