\documentclass[12pt]{article}
\usepackage{geometry}
\usepackage{amsmath}
\usepackage{amsthm}
\usepackage{amssymb}
\usepackage{mathrsfs}
\usepackage{parskip}
\usepackage{enumerate}
\usepackage{stmaryrd}
\usepackage{listings}
\usepackage{fullpage}

\begin{document}

\title{CS 348 Notes}
\author{Matthew Visser}
\date{Nov 17, 2011}
\maketitle

\section{Normal Forms}

\begin{itemize}
	\item 3\textsuperscript{rd} normal form preserves dependancies better than
		BCNF.
	\item 3\textsuperscript{rd} normal form has an algorithm that gives a
		depedancy preserving decomposition.
\end{itemize}
\section{Transactions}

\begin{itemize}
	\item Why do we need transactions?
	\item What is serializability?
	\item How do we do transactions in SQL?
	\item How do we implement transactions?
\end{itemize}

\subsection{Transactions}

\begin{itemize}
	\item Applications can group together two or more operations.
	\item Operations are assumed to take the database from a consistent state
		to another consistent state.
	\item Transactions are one or more operations in sqeuence.
\end{itemize}

There are many causes for failure, in two levels:
\begin{enumerate}
	\item program execution failure, bugs in the CPU
	\item storage failure
\end{enumerate}

Examples of problems caused by concurrency:
\begin{itemize}
	\item If we have a failure during two updates that are dependant, the
		database would be inconcistent.
	\item If we read while data is being updated in multiple tables, we could
		have an inconsistent view of the database.
	\item We could read data, do some calculation, then do an update. What if
		data is updated in between those two operations?
\end{itemize}

The ACID principle is:

\begin{tabular}{lp{0.8\textwidth}}
	\textbf{Atomic} & All operations run or none do at all.\\
	\textbf{Consistency} & Each transaction preserves database consistency.\\
	\textbf{Isolated} & Concurrent transactions do not interfere with each-other.\\
	\textbf{Durable} & Once complete, a transactions changes are permanent. We
	won't lose any data once a transaction is complete.\\
\end{tabular}

\subsection{Serializability}

\begin{itemize}
	\item The server will schedule the operations so that it seems as though
		they are run sequentially, but actually runs them concurrently if
		possible.
	\item Two operations conflict if
		\begin{enumerate}
			\item They belong to different transactions
			\item They operate on the same object, and
			\item At least one of the operations is a write
		\end{enumerate}
\end{itemize}

\subsection{Transactions in SQL}

\begin{itemize}
	\item We can abort or commit a transaction. Usually abort on some error or
		logic, commit if everything goes well.
	\item A transaction begins when a client executes a sql command.
	\item Two commands are given to terminate a transaction: \texttt{COMMIT} and
		\texttt{ROLLBACK}.
\end{itemize}



\subsection{Implementing Transactions}

\end{document}
% vim: tw=80
